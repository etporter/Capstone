\documentclass[]{article}
\usepackage[margin=1in]{geometry}

%opening
\title{Non-Visual Graphs for Blind or Low-Vision Students}
\author{Ethan Porter}

\begin{document}

\maketitle

\section{Introduction}
\subsection{Background}
Blind or Low-Vision Students or forced to live with the reality that traditional learning materials leave them at a disadvantage when compared with their sighted peers.  This is evident in the area of graphs.  We use graphs to explain abstract concepts to sighted students, but there isn't currently an effective equivalent for blind and low vision students.  Research into non-visual graph technology has previously been conducted, but we have yet to see this research made practically available to students. [Cite Klatzky, et al]
\subsection{Project Goal}
This project aims to take existing research into vibro-tactile graphs and make it practically available to students and educators.  The result should be software that allows educators to enter raw data and produce a vibro-tactile graph.  This graph can then be displayed to blind and low vision students.

\section{Design}
\subsection{Overview}
The primary concept that is pertinent to the design of the software for this project is the different needs of different clients.  This informs the design of the user interface.  Specifically, we are concerned about the needs of two different clients.

One client, the "teacher" is a sighted client.  The "student" on the other hand, is blind or low-vision.  This has a huge impact on what represents a good user interface design for the client.  For example, multiple small buttons in an area might be a viable interface for a sighted user, but a blind or low-vision user will find this difficult to navigate.  For this reason, this project will include two separate interfaces, which will be referred to as the "Input Interface" and the "Display Interface".

\subsection{Input Interface}
The Input Interface should follow straightforward and basic good practices of interface design.

\subsection{Display Interface}
The Display Interface should follow guidelines from [Klatzky, et al] as well as [Alonso, et al] for designing interfaces for blind or low-vision users.

\section{Methods}
\subsection{Testing}

\subsection{Analysis}

\section{Conclusion}

\section{Sources}

\end{document}